%% We use `subfiles' package
\documentclass[preamble.tex]{subfiles}
\begin{document}

% For lines describing combinators

\begin{figure}
%\vspace*{-1cm}
\hspace*{-1cm}
\begin{subfigure}{.6\textwidth}
\begin{hscode}[%
  columns=fullflexible, % to get well spaced subscripts
  keepspaces,           % and keep spaces
  literate=
    {_1}{{\sub{map}}}3
    {_2}{{\sub{xs}}}2,
]
entry :: [Dynamic] -> Dynamic
entry [arr_2] = toDyn (run (fromDyn arr_2))

run :: Vector Double -> Vector Double
run arr_2 = runST (init_1 arr_2)

init_1 arr_2 = do
  let len_2 = arrayLength arr_2
  let len_1 = len_2
  let ix_2 = 0
  arr_1 <- newArray len_1
  guard_1 arr_2 arr_1 ix_2 len_2

guard_1 arr_2 arr_1 ix_2 len_2 = do
  if ix_2 < len_2
    then nest_1 arr_2 arr_1 ix_2 len_2
    else done_1 arr_2 arr_1 ix_2 len_2

nest_1 arr_2 arr_1 ix_2 len_2 = do
  body_1 arr_2 arr_1 ix_2 len_2

body_1 arr_2 arr_1 ix_2 len_2 = do
  let elt_2 = readArray arr_2 ix_2
  let elt_1 = elt_2 * 100
  yield_1 arr_2 arr_1 ix_2 len_2 elt_1

yield_1 arr_2 arr_1 ix_2 len_2 elt_1 = do
  let ix_2' = ix_2 + 1
  writeArray arr_1 ix_2 elt_1
  bottom_1 arr_2 arr_1 ix_2' len_2 elt_1

bottom_1 arr_2 arr_1 ix_2 len_2 elt_1= do
  guard_1 arr_2 arr_1 ix_2 len_2

done_1 arr_2 arr_1 ix_2 len_2 = do
  result_1 <- sliceArray arr_1 ix_2
  return result_1
\end{hscode}
\end{subfigure}%
%
\begin{subfigure}{.55\textwidth}
\begin{loopcode}[%
  literate=
    {_1}{{\sub{map}}}3
    {_2}{{\sub{xs}}}2,
]






init_1/_2:
  let len_2 = arrayLength arr_2
  let len_1 = len_2
  let ix_2 = 0
  let arr_1 = newArray len_1
  goto guard_1

guard_1/_2:
  unless ix_2 < len_2 | done_1
  goto nest_1


nest_1/_2:
  goto body_1

body_1/_2:
  let elt_2 = readArray arr_2 ix_2
  let elt_1 = elt_2 * 100
  goto yield_1

yield_1/_2:
  ix_2 := ix_2 + 1
  writeArray arr_1 ix_2 elt_1
  goto bottom_1

bottom_1/_2:
  goto guard_1

done_1/_2:
  let result_1 = sliceArray arr_1 ix_2
  return result_1
\end{loopcode}
\end{subfigure}
\caption{\Haskell code (left) and \Loop code (right) generated for \code{map (* 100) xs}.}
\label{fig:times100-hs-loop}
\end{figure}


\end{document}