%% We use `subfiles' package
\documentclass[preamble.tex]{subfiles}
\begin{document}

\clearpage

\chapter{Listings}
\label{ch:listings}

\begin{figure}[ht!]
\begin{loopcode}[%
    literate=
        {_xs}{{\sub{xs}}}2
        {_r}{{\sub{r}}}1
        {_is}{{\sub{is}}}2
        {_bperm}{{\sub{bperm}}}5
]
init_bperm/_is:
  let len_is = arrayLength arr_is
  let len_xs = arrayLength arr_xs
  let len_bperm = len_is
  let ix_is = 0
  let arr_bperm = newArray len_bperm
  goto guard_bperm

guard_bperm/_is:
  unless ix_is < len_is | done_bperm
  goto body_bperm

body_bperm/_is:
  let elt_is = readArray arr_is ix_is
  -- set index to read xs at
  let ix_r = elt_is
  let elt_xs = readArray arr_xs ix_r
  let elt_bperm = elt_xs
  goto yield_bperm

yield_bperm/_is:
  ix_is := ix_is + 1
  writeArray arr_bperm ix_is elt_bperm
  goto bottom_bperm

bottom_bperm/_is:
  goto guard_bperm

done_bperm/_is:
  let result_bperm = sliceArray arr_bperm ix_is
  return (result_bperm)
\end{loopcode}

\caption{\Loop code generated for \code{bpermute xs is} where both data (\code{xs}) and indices (\code{is}) arrays are manifest and the result array is forced. Referenced in Section~\protect\ref{sec:Loops-random-access}.}
\label{fig:Loop-bpermute-complete}
\end{figure}


\begin{figure}[ht!]

\begin{subfigure}{.55\textwidth}
\begin{loopcode}[%
  literate=
    {_1}{{\sub{fold}}}3
    {_3}{{\sub{segd}}}3
    {_4}{{\sub{data}}}3,
]
init_1/_3/_4:
  let len_3 = arrayLength arr_3
  let len_4 = arrayLength arr_4
  let len_1 = len_3
  let z_1 = 0
  let ix_3 = 0
  let ix_4 = 0
  let arr_1 = newArray len_1
  goto guard_1

guard_1/_3:
  unless ix_3 < len_3 | done_1
  goto nest_1

nest_1/_3:
  let acc_1 = z_1
  let elt_3 = readArray arr_3 ix_3
  let end_3 = ix_4 + elt_3
  goto guard_4

body_1/_3:
  let elt_1 = acc_1
  goto yield_1

yield_1/_3:
  ix_3 := ix_3 + 1
  writeArray arr_1 ix_3 elt_1
  goto bottom_1

bottom_1/_3:
  goto guard_1

done_1/_3/_4:
  let result_1 = sliceArray arr_1 ix_3
  return (result_1)
\end{loopcode}
\end{subfigure}%
%
\begin{subfigure}{.45\textwidth}
\begin{loopcode}[%
  literate=
    {_1}{{\sub{fold}}}3
    {_3}{{\sub{segd}}}3
    {_4}{{\sub{data}}}3,
]
guard_4:
  unless ix_4 < end_3 | body_1
  goto nest_4

nest_4:
  goto body_4

body_4:
  let elt_4 = readArray arr_4 ix_4
  goto yield_4

yield_4:
  ix_4 := ix_4 + 1
  goto bottom_4

bottom_4:
  acc_1 := acc_1 + elt_4
  goto guard_4
\end{loopcode}
\end{subfigure}

\caption{\Loop code generated for \code{folds (+) 0 segd data} where both \code{segd} and \code{data} are manifest arrays and the result array is forced. Referenced in Section~\protect\ref{sec:Loops-segmented-reductions}.}
\label{fig:Loop-fold-s-complete}
\end{figure}


\begin{figure}[ht!]

\begin{subfigure}{.55\textwidth}
\begin{loopcode}[%
  literate=
    {_1}{{\sub{scan}}}3
    {_3}{{\sub{segd}}}3
    {_4}{{\sub{data}}}3,
]
init_1/_3/_4:
  let len_3 = arrayLength arr_3
  let len_4 = arrayLength arr_4
  let len_1 = len_4
  let z_1 = 0
  let ix_4 = 0
  let ix_3 = 0
  let arr_1 = newArray len_1
  goto guard_3

guard_3:
  unless ix_3 < len_3 | done_1
  goto nest_3

nest_3:
  let acc_1 = z_1
  let elt_3 = readArray arr_3 ix_3
  let end_3 = ix_4 + elt_3
  goto guard_1

body_3:
  goto yield_3

yield_3:
  ix_3 := ix_3 + 1
  goto bottom_3

bottom_3:
  goto guard_3

done_1/_3/_4:
  let result_1 = sliceArray arr_1 ix_4
  return (result_1)
\end{loopcode}
\end{subfigure}%
%
\begin{subfigure}{.45\textwidth}
\begin{loopcode}[%
  literate=
    {_1}{{\sub{scan}}}3
    {_3}{{\sub{segd}}}3
    {_4}{{\sub{data}}}3,
]
guard_1/_4:
  unless ix_4 < end_3 | body_3
  goto nest_1

nest_1/_4:
  goto body_1

body_1/_4:
  let elt_4 = readArray arr_4 ix_4
  let elt_1 = acc_1
  goto yield_1

yield_1/_4:
  ix_4 := ix_4 + 1
  writeArray arr_1 ix_4 elt_1
  goto bottom_1

bottom_1/_4:
  acc_1 := acc_1 + elt_4
  goto guard_1
\end{loopcode}
\end{subfigure}

\caption{\Loop code generated for \code{scanls (+) 0 segd data} where both \code{segd} and \code{data} are manifest arrays and the result array is forced. Referenced in Section~\protect\ref{sec:Loops-segmented-reductions}.}
\label{fig:Loop-scanl-s-complete}
\end{figure}


\clearpage
\section*{Segmented FilterMax in \LiveFusion}
\label{sec:FilterMax-implementation}

The following is the complete source of the program representing the FilterMax component of vectorised QuickHull. QuickHull was discussed in Section~\ref{sec:QuickHull} on page~\pageref{sec:QuickHull} with \name{Data Parallel Haskell} source given in Listing~\ref{lst:dph-qh} on page~\pageref{lst:dph-qh}.


\begin{hscode}
-- The heart of vectorised QuickHull.
-- Based on hand-vectorised version of QuickHull found in:
-- ghc/libraries/dph/dph-examples/examples/spectral/QuickHull/handvec/Handvec.hs
{-# LANGUAGE RebindableSyntax #-}
module FilterMax where

import Data.LiveFusion

import Prelude as P hiding ( map, replicate, zip, zipWith,
                             filter, fst, snd, unzip )

filterMax :: (IsNum a, IsOrd a, Elt a)
          => Term Int            -- Number of points
          -> Array Int           -- Segd
          -> Array a -> Array a  -- Points
          -> Array a -> Array a  -- Line starts
          -> Array a -> Array a  -- Line ends
          -> (Array a, Array a,  -- Farthest points
              Array a, Array a,  -- Points above lines
              Array Int)         -- New segd
filterMax npts segd xs ys x1s y1s x2s y2s
  = let -- Find distance-like measures between each point and its respective line.
        distances = calcDistances npts segd xs ys x1s y1s x2s y2s

        -- Throw out all the points which are below the line.
        (above_xs, above_ys, aboveSegd) = calcAbove npts segd xs ys distances

        -- Find points farthest from each line.
        (fars_xs, fars_ys) = calcFarthest npts segd xs ys distances
    in  (fars_xs, fars_ys, above_xs, above_ys, aboveSegd)


\end{hscode}


\clearpage
\begin{hscode}
-- | Find (relative) distances of points from line.
--
-- Each point can be above (positive distance) or below (negative distance)
-- a line as looking from the centre of the convex hull.
--
-- Corresponds to 'distances' in the original program:
-- > distances = [: distance p line | p ← points :]
calcDistances :: (IsNum a, IsOrd a, Elt a)
              => Term Int            -- Number of points
              -> Array Int           -- Segd
              -> Array a -> Array a  -- Points
              -> Array a -> Array a  -- Line starts
              -> Array a -> Array a  -- Line ends
              -> Array a             -- Relative distances from lines
calcDistances npts segd xs ys x1s y1s x2s y2s
  = zipWith6 distance xs
                      ys
                      (replicate_s npts segd x1s)
                      (replicate_s npts segd y1s)
                      (replicate_s npts segd x2s)
                      (replicate_s npts segd y2s)



-- | Compute cross product between vectors formed between a point (x,y)
--   and each of the two line ends: (x1,y1) and (x2,y2).
distance :: (IsNum a, IsOrd a, Elt a)
         => Term a -> Term a  -- Point
         -> Term a -> Term a  -- Line start
         -> Term a -> Term a  -- Line end
         -> Term a            -- Distance
distance x y x1 y1 x2 y2
  = (x1 - x) * (y2 - y) - (y1 - y) * (x2 - x)

\end{hscode}


\clearpage
\begin{hscode}
-- | Find points above the lines given distance-like measures.
--
-- Corresponds to 'above' in the original program:
-- > above = [: p | (p,c) ← zipP points distances, c > 0.0 :]
calcAbove :: (IsNum a, IsOrd a, Elt a)
          => Term Int            -- Number of points
          -> Array Int           -- Segd
          -> Array a -> Array a  -- Points
          -> Array a             -- Distances
          -> (Array a, Array a,  -- Points with positive distances
              Array Int       )  -- New Segd
calcAbove npts segd xs ys distances
  = let -- Compute selector for positive elements
        aboveTags  = zipWith (>.) distances (replicate npts 0)

        -- Compute segd for just the positive elements
        aboveSegd = count_s true segd aboveTags

        -- Get the actual points corresponding to positive elements
        (above_xs, above_ys)
                  = unzip
                  $ packByBoolTag true aboveTags
                  $ zip xs ys
 
    in  (above_xs, above_ys, aboveSegd)

\end{hscode}


\clearpage
\begin{hscode}
-- | For each line find a point farthest from that line.
--
-- Each segment is a collection of points above a certain line.
-- The array of Doubles gives (relative) distances of points from the line.
--
-- Corresponds to 'far' in the original program:
-- > far = points !: maxIndexP distances
calcFarthest :: (IsNum a, IsOrd a, Elt a)
             => Term Int            -- Number of points
             -> Array Int           -- Segment descriptor
             -> Array a -> Array a  -- Points
             -> Array a             -- Distances
             -> (Array a, Array a)  -- Points with biggest distances
calcFarthest npts segd xs ys distances
  = let -- Index the distances array, and find the indices corresponding to the
        -- largest distance in each segment
        indexed  = zip (indices_s npts segd)
                       distances
        max_ids  = fsts
                 $ fold_s maxSnd (0 .*. small) segd indexed

        small    = -999999

        -- Find indices to take from the points array by offsetting from segment starts
        ids      = zipWith (+) (indicesSegd segd)
                                max_ids
        max_xs   = bpermute xs ids
        max_ys   = bpermute ys ids

        -- We are only interested in the ones which are above the line
        -- (thus from segments containing points above line).
    in (max_xs, max_ys)


-- | Compute array containing starting indices of each segment
--   of a segment descriptor.
indicesSegd :: Array Int -> Array Int
indicesSegd = scan (+) 0


-- | Find pair with the biggest second element.
maxSnd :: (Elt a, Elt b, IsOrd b) => Term (a, b) -> Term (a, b) -> Term (a, b)
maxSnd ab1 ab2 = let b1 = snd ab1
                     b2 = snd ab2
                 in  if b1 >=. b2 then ab1
                                else ab2

\end{hscode}
%\lstinputlisting[style=haskell, float, caption={Segmented \code{FilterMap} what finds points above each line as well as farthest from each line. Reference in Section~\protect\ref{sec:Heart-of-QuickHull}.}, label=lst:AliasMap-interface]{listings/FilterMax.hs}

\end{document}