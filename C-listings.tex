%% We use `subfiles' package
\documentclass[preamble.tex]{subfiles}
\begin{document}

\clearpage

\chapter{Listings}
\label{ch:listings}

\begin{figure}[h!]
\begin{loopcode}[%
    literate=
        {_xs}{{\sub{xs}}}2
        {_r}{{\sub{r}}}1
        {_is}{{\sub{is}}}2
        {_bperm}{{\sub{bperm}}}5
]
init_bperm/_is:
  let len_is = arrayLength arr_is
  let len_xs = arrayLength arr_xs
  let len_bperm = len_is
  let ix_is = 0
  let arr_bperm = newArray len_bperm
  goto guard_bperm

guard_bperm/_is:
  unless ix_is < len_is | done_bperm
  goto body_bperm

body_bperm/_is:
  let elt_is = readArray arr_is ix_is
  -- set index to read xs at
  let ix_r = elt_is
  let elt_xs = readArray arr_xs ix_r
  let elt_bperm = elt_xs
  goto yield_bperm

yield_bperm/_is:
  ix_is := ix_is + 1
  writeArray arr_bperm ix_is elt_bperm
  goto bottom_bperm

bottom_bperm/_is:
  goto guard_bperm

done_bperm/_is:
  let result_bperm = sliceArray arr_bperm ix_is
  return (result_bperm)
\end{loopcode}

\caption{\Loop code generated for \code{bpermute xs is} where both data (\code{xs}) and indices (\code{is}) arrays are manifest and the result array is forced. Referenced in Section~\ref{sec:Loops-random-access}.}
\label{fig:Loop-bpermute-complete}
\end{figure}


\begin{figure}[h!]

\begin{subfigure}{.55\textwidth}
\begin{loopcode}[%
  literate=
    {_1}{{\sub{fold}}}3
    {_3}{{\sub{segd}}}3
    {_4}{{\sub{data}}}3,
]
init_1/_3/_4:
  let len_3 = arrayLength arr_3
  let len_4 = arrayLength arr_4
  let len_1 = len_3
  let z_1 = 0
  let ix_3 = 0
  let ix_4 = 0
  let arr_1 = newArray len_1
  goto guard_1

guard_1/_3:
  unless ix_3 < len_3 | done_1
  goto nest_1

nest_1/_3:
  let acc_1 = z_1
  let elt_3 = readArray arr_3 ix_3
  let end_3 = ix_4 + elt_3
  goto guard_4

body_1/_3:
  let elt_1 = acc_1
  goto yield_1

yield_1/_3:
  ix_3 := ix_3 + 1
  writeArray arr_1 ix_3 elt_1
  goto bottom_1

bottom_1/_3:
  goto guard_1

done_1/_3/_4:
  let result_1 = sliceArray arr_1 ix_3
  return (result_1)
\end{loopcode}
\end{subfigure}%
%
\begin{subfigure}{.45\textwidth}
\begin{loopcode}[%
  literate=
    {_1}{{\sub{fold}}}3
    {_3}{{\sub{segd}}}3
    {_4}{{\sub{data}}}3,
]
guard_4:
  unless ix_4 < end_3 | body_1
  goto nest_4

nest_4:
  goto body_4

body_4:
  let elt_4 = readArray arr_4 ix_4
  goto yield_4

yield_4:
  ix_4 := ix_4 + 1
  goto bottom_4

bottom_4:
  acc_1 := acc_1 + elt_4
  goto guard_4
\end{loopcode}
\end{subfigure}

\caption{\Loop code generated for \code{fold_s (+) 0 segd data} where both \code{segd} and \code{data} are manifest arrays and the result array is forced. Referenced in Section~\ref{sec:Loops-segmented-reductions}.}
\label{fig:Loop-fold-s-complete}
\end{figure}


\begin{figure}[h!]

\begin{subfigure}{.55\textwidth}
\begin{loopcode}[%
  literate=
    {_1}{{\sub{scan}}}3
    {_3}{{\sub{segd}}}3
    {_4}{{\sub{data}}}3,
]
init_1/_3/_4:
  let len_3 = arrayLength arr_3
  let len_4 = arrayLength arr_4
  let len_1 = len_4
  let z_1 = 0
  let ix_4 = 0
  let ix_3 = 0
  let arr_1 = newArray len_1
  goto guard_3

guard_3:
  unless ix_3 < len_3 | done_1
  goto nest_3

nest_3:
  let acc_1 = z_1
  let elt_3 = readArray arr_3 ix_3
  let end_3 = ix_4 + elt_3
  goto guard_1

body_3:
  goto yield_3

yield_3:
  ix_3 := ix_3 + 1
  goto bottom_3

bottom_3:
  goto guard_3

done_1/_3/_4:
  let result_1 = sliceArray arr_1 ix_4
  return (result_1)
\end{loopcode}
\end{subfigure}%
%
\begin{subfigure}{.45\textwidth}
\begin{loopcode}[%
  literate=
    {_1}{{\sub{scan}}}3
    {_3}{{\sub{segd}}}3
    {_4}{{\sub{data}}}3,
]
guard_1/_4:
  unless ix_4 < end_3 | body_3
  goto nest_1

nest_1/_4:
  goto body_1

body_1/_4:
  let elt_4 = readArray arr_4 ix_4
  let elt_1 = acc_1
  goto yield_1

yield_1/_4:
  ix_4 := ix_4 + 1
  writeArray arr_1 ix_4 elt_1
  goto bottom_1

bottom_1/_4:
  acc_1 := acc_1 + elt_4
  goto guard_1
\end{loopcode}
\end{subfigure}

\caption{\Loop code generated for \code{scanl_s (+) 0 segd data} where both \code{segd} and \code{data} are manifest arrays and the result array is forced. Referenced in Section~\ref{sec:Loops-segmented-data-rate-combinators}.}
\label{fig:Loop-scanl-s-complete}
\end{figure}


\end{document}